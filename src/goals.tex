\chapter{Project Goals} \label{goals}

The goal of this project is to produce a working system that meets 
requirements that will later be identified. The system will be designed and
built using the knowledge that I have built up during the course of my Open
University studies. As such it is likely to be designed using UML and
implemented using Java. The specific elements that I will deliver will be as
follows:
\begin{itemize}
  \item Output from the modelling phase of the project. This will include
  a domain model, class models etc.
  \item A database schema, designed to best practices, that will be used to
  store all of the relevant information derived from the modelling exercise.
  \item A working database implementing the database schema, populated with real
  data.
  \item An API to the database such that data can be added to the database and
  data already in the database can be retrieved. The API should use standard
  data formats for accepting data and for presenting it. This API should be
  designed in such a way that it can be extended in the future to allow for
  other operations (for example updating data already in the database, searching for data etc.)
  \item A web interface to allow viewing of the data held within the database.
\end{itemize}
Some activities will specifically be excluded from the scope of the project.
This is primarily due to time constraints but also to properly complete the
activities would require more knowledge than I have gained during the course of
my studies. If during the course of my research and development work it becomes
evident that this can be added in a trivial manner then they may later be
re-introduced into the scope of the project. These are:
\begin{itemize}
  \item API Security. It is anticipated that I will not implement access control
  on the API. A live deployment of the API will either require that
  authentication be added or an application firewall be utilised to only permit
  certain operations.
  \item Web interface to advanced operations. It is accepted that the web
  interface will only allow viewing of the data exposed by the API. In order to
  add data to the database the API will be used directly, rather than via a web
  interface.
\end{itemize}
Aside from the domain research needed for the modelling phase and associated
requirements generation, the following additional research tasks have been
identified:
\begin{enumerate}
  \item Research into the data formats that will be used to populate the
  database via the API, and the desired output formats from the API.
\end{enumerate}
The output of this research is presented in section \ref{research} beginning on
page
\pageref{research}
