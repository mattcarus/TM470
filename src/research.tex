\chapter{Research} \label{chapter:research}

\section{Data Formats}
From speaking to Katie Rhodes (British Fencing) and to other competition
organisers, it has become clear that one of two pieces of software are used to
organise fencing competitions. These are called \textit{Engarde} and
\textit{Fencing Time}. Both programs have the ability to output XML files
detailing the results of the competition \citep{engardewebsite} and
\citep{fencingtimewebsite}. XML samples from each program are supplied in
Appendix \ref{appendix-sample-fencing-xml} on page
\pageref{appendix-sample-fencing-xml}. Clearly, the two programs use different
formats.

For the output format from the API, it would be possible to re-use one
(or both) of these two formats, or a different format could be used. If a
different format is chosen then it could either be a format that I create
specifically for this system, or it could be a format published elsewhere.
Further internet-based research has indicated that there are standards in
existence for representing sports data. One such standard is known as SportsML,
the current version of which (as of Feb 2016) is SportsML-G2. This standard is
defined by the \textit{International Press Telecommunications Council (IPTC)}. I
contacted the IPTC via their developer forums to ask if fencing data could be
represented in SportsML format and the response was that it should be able to be
represented but no-one was aware of anyone currently using SportsML for fencing
data. Two interesting responses were received from the initial posting, one from
Steve Potts at the \textit{BBC} and another from Jean F\`{e}vre of
\textit{L'Agence France-Presse (AFP)}.\\
Steve Potts, in an email to me stated that the BBC did use SportsML as their
favoured format for receiving data, he also stated:
\begin{displayquote}
The BBC are interested in the concept of local generated
sports stats (compare with user generated, which it is not), and also in publishing stats for
lesser-participated sports. The effort we expend in obtaining and publishing
stats is naturally weighted towards the more popular sports, so we are
investigating opening channels (with a lower barrier to entry) of sports stats
ingest from outside our primary suppliers. Fencing isn't included in our
supplier contracts, so creating a vocabulary for it removes one barrier for us
to publish its stats. Our ideal scenario is for us to ingest a routine automated
SportsML feed from British Fencing for fixtures, results and standings to appear
across the BBC Website, Mobile App, Red Button TV, Connected TV.
\citep{pottsemail20160202}
\end{displayquote}
From this I take that the BBC would be interested in integrating with the system
I am to build in order to receive data automatically from British Fencing. This
is something the British Fencing are quite excited about! Conversely, Jean
F\`{e}vre, again in an email, states:
\begin{displayquote}
I am in SportML group since long time but I don’t like SportML too much because
it’s very complex and it must be adapted to each sport.
I work with IOC on Olympic Games since very long time (1988) and we (IOC, ASOIF,
many international federations…) have defined ODF XML format.
For Rio games we will receive ODF 2. \citep{ferveemail20160203}
\end{displayquote}
After studying at the two formats, I would dispute F\`{e}vre's assertion that
SportsML is very complex - I understood it more quickly than the ODF format.
It is clear that there are two competing formats and it makes sense for me to
decide to support primarily one format, with support for the second being made
possible if it is able to be added.
One outstanding issue I had to face before making a decision on my primary data
format was that of whether or not SportsML would actually support fencing data
or not. As suggested by members of the SportML development group, I decided to
jump in at the deep end and actually produce a valid SportsML file with fencing
data in it. This file is available in Appendix
\ref{appendix-sample-fencing-sportsml} on page
\pageref{appendix-sample-fencing-sportsml} I made a copy of this file available
to the SportsML group for their comments, which were supportive. There remained
a small issue of embedding the weapon used for a particular competition in the
format (something that is not taken care of in the core SportsML standard) but
I'm confident that a solution can be found for that as the standard is easily
extensible.
